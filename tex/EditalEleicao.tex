\documentclass[12pt]{article}
\usepackage[utf8]{inputenc}
\usepackage[T1]{fontenc}
\usepackage[brazil]{babel}
\usepackage{todonotes}
\newcommand\tab[1][1cm]{\hspace*{#1}}
\title{Edital de Eleição do Centro Acadêmico de Ciência da Computação}
\author{Centro Acadêmico Pata do Bisão}
\date{Setembro/2016}
\begin{document}
\maketitle

\section*{Disposições Gerais}
\begin{description}
	\item[Art. 1º] A eleição para a diretoria do Centro Acadêmico de Ciência da Computação -- Sorocaba -- Pata do Bisão, será realizado no período do dia 12 ao dia 16 do mês de setembro de 2016, das 12:00 às 14:00, no prédio ATLab (Prédio Roxo) nos corredores dos laboratórios de ensino da computação. 
	\item[Art. 2º] A eleição será feita através do voto direto, secreto e universal.
	\item[Paragrafo Único] Poderão  votar  e  ser  votados  todos  os  alunos regularmente matriculados  no  curso  de Ciência da Computação. 
	\item[Art. 3º] A participação nesta eleição dará através do registro de chapas para a diretoria do Centro Acadêmico de Ciência da Computação -- Sorocaba -- Pata do Bisão.
	\item[Art. 4º] Será eleita a chapa que obtiver a maioria simples de voto, não computados os votos nulos.
\end{description}

\section*{Do Registro de Chapas}
\begin{description}
	\item[Art. 5º] As Chapas poderão efetuar seus registros no período de 29 de Agosto a 3 de Setembro no ano de 2016, das 08:00 às 18:00, com qualquer membro da comissão eleitoral.
	\item[Art. 6º] O pedido de registro deve ser feito com o nome da chapa, uma lista dos componentes da chapa com a devida indicação do cargo que o mesmo ocupará na diretoria da entidade, junto com o registro acadêmico (RA), ano de entrada, apelido (se houver) e as respectivas assinaturas, dando ciência da sua inscrição no pleito.
Os cargos que compõem a diretoria do Centro Acadêmico de Ciência da Computação - Sorocaba - Pata do Bisão são:
\begin{itemize}	
	\item Presidente;
	\item Vice-presidente;
	\item Coordenador de Finanças;
	\item Vice-coordenador de finanças;
	\item Secretário;
        \item Diretor de Comunicação;
	\item Diretor Sociocultural;
\end{itemize}
	\begin{description}
		\item[Parágrafo 1º] Cargos como Presidência e Finanças não podem ter o ano de formação no período de 2016/2 e 2017/1;
		\item[Parágrafo 2º] Membros da comissão eleitoral não podem estar afiliados a nenhuma das chapas candidatas
	\end{description}
	\item[Art. 7º] A comissão eleitoral divulgará no dia 3 de Setembro de 2016 a lista das chapas cujas inscrições foram deferidas e indeferidas.
\end{description}

\section*{Da Votação}
\begin{description}
	\item[Art. 8º] Votação será feita por um sistema manual com células de papel confeccionadas pela comissão eleitoral.
	\item[Art. 9º] Somente poderão votar estudantes regularmente matriculados, de acordo com a listagem fornecida pela Coordenação do curso de Ciência da Computação.
	\item[Art. 10º] Para votar, o eleitor deverá identificar-se por algum documento que contenha foto. O eleitor então assinará seu nome na lista de verificação, e assinalará na cédula, rubricada pelo mesário, no campo correspondente à chapa de sua preferência. Observado esses procedimentos o eleitor conclui o processo depositando sua cédula na urna.
\end{description}

\section*{Dos Mesários}
\begin{description}
	\item[Art. 11º] Os mesários serão os membros da comissão eleitoral ou pessoas escolhidas por estes que não integrem nenhuma das chapas.
\end{description}
\section*{Da Apuração}
\begin{description}
	\item[Art. 12º] A apuração iniciará logo após o termino da votação e será públicada na página do Centro Acadêmico de Ciência da Computação -- Sorocaba.
	\item[Art. 13º] Serão considerados nulos todos os votos que contenham inscrições que não deixem evidente a opção do eleitor por alguma das chapas, bem como aquelas cédulas que não estiverem rubricadas por pelo menos, um membro da Mesa Receptora.
\end{description}

\section*{Disposições Finais}
\begin{description}
	\item[Art. 14º] A comissão eleitoral deverá organizar um debate entre as chapas, caso exista mais de uma chapa inscrita, caso contrário, deverá ser organizado uma atividade com a apresentação do papel do Centro Acadêmico junto com as propostas da chapa.
	\item[Art. 15º] Caso não tenha nenhuma chapa inscrita, a comissão eleitoral deverá realizar mais uma vez (e apenas uma vez) o processo eleitoral;
	\item[Art. 16º] Os casos omissos serão resolvidos pela comissão eleitoral.
\end{description}
\end{document}
